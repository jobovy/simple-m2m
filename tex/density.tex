\documentclass[12pt,preprint]{aastex}
\usepackage{amssymb,amsmath}
%\usepackage{color,hyperref}
% hypertex insanity
%\definecolor{linkcolor}{rgb}{0,0,0.25}
%\hypersetup{
%  colorlinks=true,        % false: boxed links; true: colored links
%  linkcolor=linkcolor,    % color of internal links
%  citecolor=linkcolor,    % color of links to bibliography
%  filecolor=linkcolor,    % color of file links
%  urlcolor=linkcolor      % color of external links
%}
\newcounter{address}
\setcounter{address}{1}
%\usepackage{sidecap}
\setlength{\emergencystretch}{2em}%No overflowing references
\newcommand{\ie}{i.e.}
\newcommand{\etal}{et al.}
\newcommand{\dd}{\mathrm{d}}
\newcommand{\eg}{e.g.}
\newcommand{\eqnname}{equation}
\newcommand{\equationname}{\eqnname}
\newcommand{\figurenames}{\figurename s}

\newcommand{\dens}{\ensuremath{\nu_*}}
\newcommand{\zsun}{\ensuremath{z_\odot}}
\newcommand{\gaia}{\emph{Gaia}}
\newcommand{\tgas}{\emph{TGAS}}
\newcommand{\ra}{\ensuremath{\alpha}}
\newcommand{\dec}{\ensuremath{\delta}}
\newcommand{\plx}{\ensuremath{\varpi}}
\usepackage{yfonts}
\newcommand{\essf}{\ensuremath{\textswab{S}}}

\newcommand{\pc}{\ensuremath{\,\mathrm{pc}}}

\begin{document}

\title{Determining stellar density laws with \tgas}
\author{Jo~Bovy\altaffilmark{1,2,3}}
\altaffiltext{\theaddress}{\label{1}\stepcounter{address}
  Department of Astronomy and Astrophysics, University of Toronto, 50
  St.  George Street, Toronto, ON, M5S 3H4, Canada;
  bovy@astro.utoronto.ca~}
\altaffiltext{\theaddress}{\label{2}\stepcounter{address} 
  Center for Computational Astrophysics, Flatiron Institute, 162 5th Ave, New York, NY 10010, USA}
\altaffiltext{\theaddress}{\label{3}\stepcounter{address}
  Alfred~P.~Sloan~Fellow}

\begin{abstract} 
  This note discusses how to observationally determine stellar density
  laws from the \tgas\ data, taking into account the \tgas\ selection
  function.
\end{abstract}

\section{Introduction}

The general problem under consideration here is how to determine
stellar density laws $\dens(X,Y,Z)$ of stellar populations in the
Milky Way from \tgas\ observations of stars providing precise
$(\ra,\dec,\plx,G)$. These notes follow the discussion in
\citet{Bovy16a}, which the reader should familiarize themselves with,
although the discussion below is largely self-contained.

\section{Likelihood-based density modeling}

We are interested in $\dens(X,Y,Z)$, where $(X,Y,Z)$ is a set of
cartesian coordinates related by a rotation to the spherical
coordinates ($\ra,\dec,D=1/\plx$) in which observations are
made\footnote{We write distance $D=1/\plx$ as a shorthand here; if
  $\sigma_\plx$ is large, care and prior information must be used to
  infer $D$ from $\plx$.}. Following \citet{Bovy16a}, we determine
$\dens(X,Y,Z)$ from a population of stars that is not complete in any
(simple) geometric sense and thus need to take the selection function
into account. We assume that the selection function is a function of
(a) position on the sky (\ra,\dec), (b) an apparent magnitude (here
\gaia\ $G$), and potentially (c) a color $c$. We further assume that
the absolute magnitude $M_G$ of stars in the population under
investigation can be (closely enough) determined from $G$ and $c$,
such that we can connect the underlying stellar density in $(X,Y,Z)$
to the distribution in apparent magnitude.

To determine $\dens(X,Y,Z)$ we use a likelihood approach that models
the full rate function $\lambda(O|\theta)$ that gives the number of
stars as a function of all observables $O$ of interest for a set of
model parameters $\theta$. These observables $O$ are in this case
$(\ra,\dec,D,G,c)$---we will use $(X,Y,Z)$ and $(\ra,\dec,D)$
interchangeably because they are related by coordinate transformation,
but will keep track of the Jacobian---and we can write
\begin{equation}
\begin{split}
  \lambda(O|\theta) & = \lambda(\ra,\dec,D,G,c)\,,\\
  &  = \dens(X,Y,Z|\theta)\,D^2\,\cos \dec\,\rho(M_G,c|X,Y,Z)\,S(\ra,\dec,G,c)\,,
\end{split}
\end{equation}
where we have assumed that the model parameters only affect
$\dens$. In this decomposition, the factor $D^2\,\cos \dec$ comes from
the Jacobian of the transformation between $(\ra,\dec,D)$ and
$(X,Y,Z)$, $S(\ra,\dec,G,c)$ is the \tgas\ selection function, and
$\rho(M_G,c|X,Y,Z)$ gives the density distribution of $(M_G,c)$, which
may be a function of position ($X,Y,Z$), of the stellar population. 

An observed set of stars indexed by $i$ forms a Poisson process with
the likelihood $\mathcal{L}(\theta)$ of the parameters $\theta$
describing the density law given by
\begin{equation}
\begin{split}
  \ln \mathcal{L}(\theta) & = \sum_i \ln \lambda(O_i|\theta) -\int \dd O
  \lambda(O|\theta)\,,\\ & = \sum_i \ln \dens(X_i,Y_i,Z_i|\theta)\\
  & \quad -\int \dd D\, D^2\dd \ra \dd \dec
    \,\cos\dec\,\dens(X,Y,Z|\theta)\int \dd G \dd c
      \,\rho(M_G,c|X,Y,Z)\,S(\ra,\dec,G,c)\,,\\
\end{split}
\end{equation}
where in the second line we have dropped terms that do not depend on
$\theta$.  As in \citet{Bovy16a} we simplify this expression by
defining the \emph{effective selection function} $\essf(\ra,\dec,D)$
defined by
\begin{equation}
  \essf(\ra,\dec,D) \equiv \int \dd G \dd c \,\rho(M_G,c|X,Y,Z)\,S(\ra,\dec,G,c)\,,
\end{equation}
where we use that $5\log_{10}\left(D/10\pc\right) =
G-M_G$\footnote{This assumes that there is no extinction due to
  dust. If there is extinction $A_G(\ra,\dec,D)$, this should be taken
  into account in this equation.}. The ln likelihood then becomes
\begin{equation}
  \ln \mathcal{L}(\theta) = \sum_i \ln \dens(X_i,Y_i,Z_i|\theta)
  -\int \dd D\, D^2\dd \ra \dd \dec
    \,\cos\dec\,\dens(X,Y,Z|\theta)\,\essf(\ra,\dec,D)\,.\\
\end{equation}
Unlike the selection function $S(\cdot)$ which is a function of the
survey's operations only (which parts of the sky were observed, for
how long, \ldots), the effective selection function $\essf(\cdot)$ is
a function of both the survey operations \emph{and} the stellar
population under investigation. Its usefulness derives from the fact
that it encapsulates all observational effects due to selection and
dust obscuration and turns the inference problem into a purely
geometric problem.

To determine the best-fit parameters $\hat{\theta}$ of a parameterized
density law $\dens(X,Y,Z|\theta)$ one has to optimize the ln likelihood
given above. This ln likelihood can be marginalized analytically over
the overall amplitude of the density (the local normalization if you
will); this is discussed in \citet{Bovy16a} and similar expressions
would apply here.

\section{Non-parametric binned density laws}

Now suppose that one wants to determine the density $\dens(X,Y,Z)$ of
a stellar population in a set of bins in $(X,Y,Z)$. The bins are given
by a set $\{\Pi_k\}_k$ of rectangular functions that are equal to one
within the domain of the bin and zero outside of it. The domain can
have an arbitrary shape, but typically this would be an interval in
each of $X$, $Y$, and $Z$ or perhaps in $R$ and $Z$, where $R$ is the
Galactocentric radius. We can then write the density as
\begin{equation}
  \dens(X,Y,Z|\theta) = \sum_k n_k\,\Pi_k(X,Y,Z)\,,
\end{equation}
where $\theta\equiv\{n_k\}_k$ is a set of numbers that give the
density in each bin and that therefore parameterizes the density law.

The ln likelihood then becomes
\begin{equation}
\begin{split}
  \ln \mathcal{L}(\{n_k\}_k) & = \sum_i \ln \sum_k n_k\,\Pi_k(X_i,Y_i,Z_i)
  -\int \dd D\, D^2\dd \ra \dd \dec
    \,\cos\dec\,\sum_k n_k\,\Pi_k(X,Y,Z)\,\essf(\ra,\dec,D)\,,\\
    & = \sum_k \left[ N_k \ln n_k - n_k \int \dd D\, D^2\dd \ra \dd \dec
    \,\cos\dec\,\Pi_k(X,Y,Z)\,\essf(\ra,\dec,D)\right]\,,\\
\end{split}
\end{equation}
where $N_k$ is the number of points $i$ in the observed set that fall
within bin $k$. We can maximize this likelihood for each $n_k$
analytically and find best-fit $\hat{n}_k$
\begin{equation}
  \hat{n}_k = \frac{N_k}{\int \dd D\, D^2\dd \ra \dd \dec
    \,\cos\dec\,\Pi_k(X,Y,Z)\,\essf(\ra,\dec,D)}\,.
\end{equation}
The denominator in this expression is known as the \emph{effective
  volume}. Using the same symbol $\Pi_k$ to denote the integration
volume and using $x = (X,Y,Z)$ and $(\ra,\dec,D)$ interchangeably because
they are related through coordinate transformation (as we've been
doing all along), this can be written as the following simple
expression
\begin{equation}
  \hat{n}_k = \frac{N_k}{\int_{\Pi_k} \dd^3 x\,\essf(\ra,\dec,D)}\,.
\end{equation}
Thus, the effective volume is the spatial integral of the effective
selection function. This expression makes sense, because for a
complete sample $\essf(\ra,\dec,D) = 1$ and this expression simplifies
to the number divided by the volume of the bin, the standard way to
compute a number density.

From the second derivative of the ln likelihood, we find the
uncertainty on the $\hat{n}_k$
\begin{equation}
  \sigma_{\hat{n}_k} = \frac{\hat{n}_k}{\sqrt{N_k}}\,,
\end{equation}
which again makes sense.

\section{Application to \tgas}

To apply this formalism to the \tgas\ data, one requires the following
\begin{itemize}
  \item $S(\ra,\dec,G,c)$: This can be obtained by comparing the
    \tgas\ catalog to the 2MASS catalog. In principle this is best
    done in a pixelized representation of the sky ($\ra,\dec$), \eg,
    through use of HEALPix. The best form of the dependence on $G$ may
    be a smooth function or a binned representation. Whether the
    selection function has any significant dependence on a color $c$
    for samples of interest is unknown to me.

    If a cut on parallax uncertainty is made, then the selection
    function needs to be determined by comparing the sample resulting
    from the cut with the underlying 2MASS sample. Because parallax
    uncertainty depends on $c$, this will almost certainly introduce a
    dependence on $c$, although this may be minimized by restricting
    the color range. To allow for different parallax-uncertainty cuts
    (and other \tgas\ cuts), we would therefore ideally be able to
    determine the selection function on the fly. However, the
    denominator (the 2MASS number counts) could be cached on a fine
    HEALPix grid (HEALPix is best, because TGAS has a HEALPix index
    that can then be used to quickly determine the selection
    function).
  \item $\essf(\ra,\dec,D)$: The calculation of $\essf(\ra,\dec,D)$
    requires the density $\rho(M_G,c|X,Y,Z)$ [in addition to
      $S(\ra,\dec,G,c)$]. This can be obtained most easily from a
    Monte Carlo sampling of the \tgas\ sample of interest. The
    \tgas\ sample provides a sample $(M_{G,j},c_j)$ and the integral
    over $\rho(M_G,c)$ in the effective selection function can thus be
    performed through Monte Carlo integration using this sample:
    \begin{equation}
      \essf(\ra,\dec,D) \approx \sum_j S(\ra,\dec,5\log_{10}\left(D/10\pc\right)+M_{G,j},c_j)\,.
    \end{equation}
    The sample $(M_{G,j},c_j)$ can be restricted to a certain spatial
    region (\eg, $\Pi_k$) if there is any concern that the density
    $\rho(M_G,c|X,Y,Z)$ depends significantly on $(X,Y,Z)$, but this
    is unlikely.
\end{itemize}

\begin{thebibliography}{}
%\bibitem[Bovy(2015)]{Bovy15a}
%  Bovy, J. 2015, \apjs, 216, 29
\bibitem[Bovy \etal(2016)]{Bovy16a} Bovy, J., Rix, H.-W., Green,
  G.~M., Schlafly, E.~F., \& Finkbeiner, D.~P.\ 2016, \apj, 818, 130
\end{thebibliography}

\end{document}

